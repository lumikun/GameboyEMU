%%
% Please see https://bitbucket.org/rivanvx/beamer/wiki/Home for obtaining beamer.
%%
\documentclass{beamer}
\usepackage[utf8]{inputenc}

\title{Prakses Aizstāvēšana}
\subtitle{Gameboy Emulātors}
\author{Māris Muižnieks}


\begin{document}
\begin{frame}
	\maketitle
\end{frame}
\begin{frame}
\frametitle{Kas ir Emulātors?}
	\begin{itemize}
		\item Programma kas imitē kādas citas programas vai ierīces darbību.
		\item Elektroniskās iekārtas simulācīja.
		\item Konsoles simulācīja
		\item Citu arhitektūru procesoru simulācīja
	\end{itemize}
\end{frame}
\begin{frame}
	\frametitle{Kas ir Gameboy}
	\begin{itemize}
		\item Portablā Spēļu konsole 
		\item Izveidotājs ražotājs Nintendo, Japāna
		\item Izlaista 21. Aprilī 1989 gadā
	\end{itemize}
\end{frame}
\begin{frame}
	\frametitle{Kā emulēt Gameboy}
	\begin{itemize}
		\item Ielādēt Kartridžas datus/ROM failu
		\item Procesora Emulācīja/Simulēšana
		\item Saziņas kanālu simulācīja (BUS, I/O, u.t.t.)
		\item Grafikas procesora emulācīja
	\end{itemize}
\end{frame}
\begin{frame}
	\frametitle{Secinājumi}
	\begin{itemize}
		\item Darbs noteikti ir bijis sarežģīts
		\item Iegūtas jaunas zināšanas, Funkciju rādītāji, bitu līmeņa operācījas utt.
		\item Nebija pārdomāta Koda organizācīja
		\item Koda refaktūra
		\item Automātiskā Make/Build sistēma
	\end{itemize}
\end{frame}
\begin{frame}
	\frametitle{Neliela Demonstrācīja.}
\end{frame}
\end{document}
